\chapter{Gatling - Testing Framework}

%https://gatling.io/

\section{Einführung}

Was ist Gatling?

\subsection{Einsatzgebiete}

%TODO

\begin{itemize}
    \item Vorhersage von Bottlenecks in Anwendung 
    \item Live gehen ohne Angst dass Anwendung mit Last nicht klar kommt
    \item Benutzer Bedienung verbessern -- Latenz Zeiten verringern = schnelleres Feedback für den User
    \item langfristig Verbesserung / Optimierung der Software Architektur  --> wie gut skaliert die Anwendung?

\end{itemize}

\section{DSL}

%https://gatling.io/docs/current/cheat-sheet/

\subsection{Wichtige Befehle}

%Tabelle mit wichtigsten Befehlen zeigen


\section{Metriken}

Welche Metriken sind verfügbar ?
Wie werden diese gesammelt?
Wie werden diese interpretiert?

\section{Continous Integration}

Wie kann Gatling in den bestehenden / zukünftigen Continous Integration Prozess eingebunden werden?

Beispielhaft Jenkins auf Hersteller-seite vorgestellt.


\section{Tests}

Code Beispiel eines Tests zeigen

Unterschiede Lasttests zu Unit / Integrationtests usw.
%https://jaxenter.de/mit-dem-testen-von-anwendungen-ist-es-so-eine-last-erst-recht-mit-lasttests-27564



Allgemein Alternative mit JMeter
%https://www.heise.de/developer/artikel/Last-und-Performance-Tests-mit-JMeter-oder-Gatling-3648505.html

JMeter vs Gatling
%https://octoperf.com/blog/2015/06/08/jmeter-vs-gatling/


\section{Zwischenfazit Lasttests}

Performance oder Lasttests sind notwendig um eine stabile Anwendung bzw. API zu erreichen.
Langfristig wird sich die Architektur bzw. der Programmierprozess verbessern wenn die Ergebnisse der Tests analysiert, ausgewertet und auch für zukünftige Entwicklung berücksichtigt werden.


